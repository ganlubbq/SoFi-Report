There are also a number of more sophisticated algorithms
that provide better direction estimation results, be it
by providing better resolution, noise imunity or the
ability to separate multiple signal sources occupying
the same frequency range. \\

On of the earliest of these algorithms is MUSIC
(MUltiple Signal Classification), first published in
1979 by Ralph O. Schmidt \cite{schmidt1979}. \\

Unlike the algorithm used in this thesis MUSIC
does not perform a transformation to the frequency-domain
but performs the direction estimation directly
in the time-domain. \\


\begin{subchapter}{Signal compositon}
  MUSIC assumes, that the analyzed spectrum contains
  $I$ discrete, narrowband signal sources overlayed by
  uncorrelated noise.

  A delay in one of these narrowband signals can be approximated
  by a phase shift, as the signal's behaviour closely resembles
  that of a sinusoid \cite[p. 4]{girdmusic}.

  In a multi-antenna arrangement the same signal $s_{i}(t)$ will be
  received by the antennas with different amounts of delay.
  Given, that the delays can be modeled by phase shifts,
  the received signals $\vec{x}(t)$ can be expressed by the multiplication
  of the input signal by a vector $\vec{a_{i}}(\theta)$ that models the
  relative phase-shifts.

  \begin{equation}
    \vec{x}(t) =
    \begin{pmatrix}
      e^{-j \cdot \Delta\varphi_{i,1}} \\
      e^{-j \cdot \Delta\varphi_{i,2}} \\
      e^{-j \cdot \Delta\varphi_{i,3}} \\
      e^{-j \cdot \Delta\varphi_{i,4}}
    \end{pmatrix} s_{i}(t) = \\
    \vec{a_{i}}(\theta) s_{i}(t)
  \end{equation}

  The phase-shifting vector $\vec{a_{i}}(\theta)$ is called the
  steering vector, it depends on the geometry of the antenna array
  and the direction of arrival of the signal $i$ \cite{chengokeda2010}. \\

  For multiple signals the model can be expanded by
  introducing a signal vector $\vec{s}(t)$ consisting
  of the $I$ narrowband signals and a steering matrix
  $A$ consisting of the separate steering vectors.

  \begin{equation}
    \vec{x}(t) =
    \begin{pmatrix}
      e^{-j \cdot \Delta\varphi_{1,1}} & e^{-j \cdot \Delta\varphi_{2,1}} & e^{-j \cdot \Delta\varphi_{3,1}} \\
      e^{-j \cdot \Delta\varphi_{1,2}} & e^{-j \cdot \Delta\varphi_{2,2}} & e^{-j \cdot \Delta\varphi_{3,2}} \\
      e^{-j \cdot \Delta\varphi_{1,3}} & e^{-j \cdot \Delta\varphi_{2,3}} & e^{-j \cdot \Delta\varphi_{3,3}} \\
      e^{-j \cdot \Delta\varphi_{1,4}} & e^{-j \cdot \Delta\varphi_{2,4}} & e^{-j \cdot \Delta\varphi_{3,4}}
    \end{pmatrix}
    \cdot
    \begin{pmatrix}
      s_{1}(t) \\
      s_{2}(t) \\
      s_{3}(t)
    \end{pmatrix}
    = \\
    \bm{A} \vec{s}(t)
  \end{equation}

  In an actual system the received signals $\vec{x}(t)$ will also
  each be superimposed by noise $n_{1}$…$n_{4}$.

  \begin{equation}
    \vec{x}(t) =  \bm{A} \vec{s}(t) + \vec{n}(t)
  \end{equation}

  With full knowledge of the steering-matrix $\bm{A}$ the
  directions of arrival could be inferred for every received
  signal. Unfortunately the noise vector $\vec{n}$ and the
  original signals $\vec{s}$ are unkown at the receiver prohibiting
  a direct calculation of the coefficients of $\bm{A}$.
\end{subchapter}

\begin{subchapter}{Signal decomposition}
  To separate the noise and payload subsignals
  MUSIC makes use of a correlation Matrix,
  that estimates the correlation between the received
  signals, and principal component analysis to
  extract the main sources of correlation \cite{vibergottersten1991}. \\

  A MUSIC implementation keeps track of the received signal
  correlations in a correlation matrix $R$.

  \begin{equation}
    \bm{R} =
    \begin{pmatrix}
      1          & \rho_{1,2} & \rho_{1,3} & \rho_{1,4} \\
      \rho_{2,1} &          1 & \rho_{2,3} & \rho_{2,4} \\
      \rho_{3,1} & \rho_{3,2} &          1 & \rho_{3,4} \\
      \rho_{4,1} & \rho_{4,2} & \rho_{4,3} &          1 \\
    \end{pmatrix} , \quad \text{where}
    \rho_{i,j} = \frac{\mathrm{Cov}(x_{i}, x_{j})}{\sqrt{\mathrm{Var}(x_{i})\mathrm{Var}(x_{j})}}
    \quad \text{\cite{wikicorrelation}}
  \end{equation}

  As the $R$ is non-negative definite \cite[p. 9]{girdmusic}
  it can be transformed into a diagonal form using
  principal component anaylsis by finding the eigenvectors and -values
  \cite[p. 325]{bronstein2016}.

  \begin{equation}
    \bm{R} = \bm{U} \bm{D} \bm{U}^{H}
  \end{equation}

  Where the diagonal elements of $\bm{D}$
  are the eigenvalues of $\bm{R}$ and $\bm{U}$
  consists of its normalized eigenvectors.
  $\bm{D}$ and $\bm{U}$ should be sorted in a way,
  that the eigenvalues are sorted in descending order. \\

  For a number of signals $I$ smaller than the
  number of receivers $N$, the first $I$ eigenvalues
  will correspond to the signal strengths of the signals
  and the remaining $N - I$ will be small as they
  mostly correspond to the correlation of the noise. \\

  This means, that as long as the number of signals is
  known and smaller than the number of receivers the
  diagonal matrix $\bm{D}$ can be separated into a part
  corresponding to the signals and a part corresponding to
  the noise.

  MUSIC uses informations from the noise-subspace to extract
  DOA informations \cite{madisetti2010}, other algorithms
  like e.g. ESPRIT use the signal-subspace for DOA-estimation.

  % TODO: continue describing orthogonality

\end{subchapter}
