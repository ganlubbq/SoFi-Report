In addition to frequency locking, that is performed in
hardware, additional synchronization steps are
performed. These steps are now presented in the
order they are performed. \\

\begin{subchapter}{Sample-exact time compensation}
  The first offset that needs to be compensated stems
  from the fact, that the SDR-dongles do not start
  recording samples at the same time.
  This effect is demonstrated in figures
  \ref{img:samples_actual_time} to \ref{img:samples_synchronized_time}. \\

  Figure \ref{img:samples_actual_time} shows the
  times at which samples were taken and put into the
  sample-buffer of the corresponding receiver, along
  with their index in the sample buffer.  \\

  In a more realisitic diagram the differences between the
  times at which the first samples are recored would be
  a lot bigger compared to the sampling rate, as the
  differences in startup time are usually in the range
  of hundreds of milliseconds and the sampling rates are
  in the order of a couple kiloherz to a few megaherz. \\

  \figurizefile{diagrams/samples_actual_time.tex}
               {img:samples_actual_time}
               {Samples and the actual time they were taken}
               {0.5}{ht}

  Upon reading the samples from the device-buffers
  the information considering the absolute aquisition time is
  lost. To a program reading the \acrshort{adc} data the
  stream of samples thus look like in figure \ref{img:samples_observed_time}. \\

  \figurizefile{diagrams/samples_observed_time.tex}
               {img:samples_observed_time}
               {Sample times when read from the device}
               {0.5}{ht}

  In order to be able to analyze phase differences between
  the signals later on, the receivers have to be synchronized to
  a accuracy of within a couple sampling times.

  The necessary synchronization is implemented by discarding
  all samples in the device buffers that were recorded
  before every device was recording samples.

  This step is demonstrated in figure \ref{img:samples_synchronized_time}.
  Receiver two was the last receiver to be started,
  thus every sample recorded before receiver two was aquirirng
  data was discarded.

  \figurizefile{diagrams/samples_synchronized_time.tex}
               {img:samples_synchronized_time}
               {Samples after sample-exact synchronization}
               {0.5}{ht}

  In order to perform this sample-exact synchronization
  one of the receivers is used as a reference.
  The other receivers are synchronized to this reference
  by calculating the cross-correlation of the samples between
  the receiver and the reference.
  And discarding samples until the cross-correlation peak
  occurs at a time-offset of zero. \\

  To facilitate calculating the cross correlation over
  more than $\SI{100}{\kilo\relax}$ samples, a \acrlong{fft}
  is used to transfer the calculation into the frequency domain.
  The transformed cross-correlation calculation can be
  seen in equation \ref{eq:fast_cross_correlation}, where
  $\mathcal{F}$ denotes the fourier transformation,
  $\mathcal{F}^{-1}$ is the inverse fourier transformation
  and an asterisk marks the complex conjugate.

  \begin{equation}
    \label{eq:fast_cross_correlation}
    f \star g = \mathcal{F}^{-1}\left(
      \mathcal{F}\left( f \right)^{\ast}
      \cdot
      \mathcal{F}\left( g \right)
    \right)
  \end{equation}

  The calculation above actually yields a
  cyclic cross-correlation \cite[p. 329]{kammeyer2012},
  so some precautions have to be taken when
  determining the position of the peak for
  synchronization. \\

  This compensation step only has to be performed
  once upon startup, as the sampling clocks are
  derived via \glspl{pll} from the same base clock,
  due to the hardware modifications.
  Thus, as long as the \glspl{pll} stay locked,
  the \glspl{adc} will produce samples at the
  same rate and will not drift apart.
\end{subchapter}

\begin{subchapter}{Sub-sample time compensation}
  As figure \ref{img:samples_actual_time} shows, the
  differences in sample time between the receivers do not
  have to be integer multiples of the sample aquisition durations. \\

  This implies, that the sample-exact time compensation above
  can not provide complete sample timing compensation, as
  a fractional difference might still remain after synchronization.

  Compensating these offsets in the time domain would
  require fractionally resampling the incoming data streams
  in order to be able to shift the signals by less than
  one sample. \\

  Instead, the input signals are now transferred to the frequency domain
  frequency domain. Figure \ref{img:preprocessing_chain}
  shows the steps that are performed for every receiver pair prior to
  the sub-sample synchronization. \\

  \figurizefile{diagrams/preprocessing_chain.tex}
               {img:preprocessing_chain}
               {Block diagram of the preporcessing chain for two receivers}
               {0.7}{ht}

  The input streams from the receivers are first delayed
  by the number of samples determined in the previous
  synchronization step, so that samples recoreded at the
  same time are in the same position in the steams. \\

  Next the samples are transformed to the frequency domain
  using \glspl{fft}.
  For one of the output vectors of the two \glspl{fft} the
  element-wise complex-conjugate is calculated. \\

  The resulting frequency-domain vectors are then
  multiplied element-wise. This complex conjugate multiplication
  yields a vector of complex numbers which contain the
  phase differences for the observed frequencies in
  their argument. As can be seen in equation
  \ref{eq:multiply_cplx_conj}. \\

  \begin{equation}
    \label{eq:multiply_cplx_conj}
    \begin{split}
      \bm{z}_\text{1} \cdot \bm{z}_\text{2}^{\ast}
      & = \abs{\bm{z}_\text{1}} e^{\arg(\bm{z}_\text{1})}
        \cdot
        \abs{\bm{z}_\text{2}} e^{-\arg(\bm{z}_\text{2})} \\
      & = \abs{\bm{z}_\text{1}} \cdot \abs{\bm{z}_\text{2}}
        e^{\arg(\bm{z}_\text{1}) - \arg(\bm{z}_\text{2})} \\
      & = A^{2} \cdot e^{\Delta \varphi}
    \end{split}
  \end{equation}

  The effect of this operation on the same complex signal
  shifted in phase is shown in figure \ref{img:complex_conjugate_mul} \\

  In order to reduce the processing load on the following stages,
  the resulting complex values are then low pass filtered by
  averaging and subsequently decimated.

  \figurizefile{diagrams/complex_conjugate_mul.tex}
               {img:complex_conjugate_mul}
               {Complex conjugate multiplication of phase shifted signals}
               {0.6}{ht}

  As these operations are performed for every pair
  of antennas in the array, $n_\text{ant}$ input
  signals are mapped to $\left(n_\text{ant}-1\right)!$
  output signals.
  The rig at hand uses
  four antennas that produce six difference-signal streams. \\

  Figure \ref{img:annotated_fft_mag} shows the mean
  absolute values of the resulting complex output
  vectors for an actual measurement in the FM-Radio band.
  The peaks in the diagram indicate
  the presence of a transmitter at the corresponding frequency.

  \figurizefile{diagrams/annotated_fft_mag.tex}
               {img:annotated_fft_mag}
               {Mean magnitude spectrum for four receivers}
               {1}{H}

  Figure \ref{img:annotated_fft_phase_orig} shows the phases
  of the six complex output vectors. When correlating the
  diagram with figure \ref{img:annotated_fft_mag} one can already
  see some bumps in the phase differences where there are strong transmitters. \\

  \figurizefile{diagrams/annotated_fft_phase_orig.tex}
               {img:annotated_fft_phase_orig}
               {Raw phase differences for four receivers}
               {1}{ht}

  One can also see a rather steep sloping of the phases,
  this is due to the sub-sample timing offsets between the
  \glspl{adc}, to compensate them a linear correction is applied
  to the phase.
  The resulting sample time compensated phase diagram is shown in
  figure \ref{img:annotated_fft_phase_timesync}.

  \figurizefile{diagrams/annotated_fft_phase_timesync.tex}
               {img:annotated_fft_phase_timesync}
               {Phase differences after sample time offset compensation}
               {1}{H}
\end{subchapter}

\begin{subchapter}{Tuner phase offset compensation}
  Figure \ref{img:annotated_fft_phase_timesync} shows a flattened
  phase diagram, but there are still constant offsets.
  These offsets are caused by different phases of the
  \acrshort{losig}-signals used to
  mix down the RF-signals in the tuners. \\

  Figure \ref{img:annotated_fft_phase_comp} shows the phases
  after compensating for these offsets. Frequency ranges where
  the noise is dominant show no phase differences between
  the receivers, while ranges with active transmitters contain
  visible bumps. When zoomed in like in figure \ref{img:annotated_fft_phase_zoom}
  these bumps become a lot more prominent. \\

  \figurizefile{diagrams/annotated_fft_phase_comp.tex}
               {img:annotated_fft_phase_comp}
               {Completely compensated phase diagram}
               {1}{H}

  \figurizefile{diagrams/annotated_fft_phase_zoom.tex}
               {img:annotated_fft_phase_zoom}
               {Completely compensated phase diagram with different scaling}
               {1}{H}


  The relative phase differences will later be used to
  determine the direction of arrival of the incoming signals.
\end{subchapter}
