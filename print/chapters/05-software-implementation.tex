The implementation is split into two software parts,
one implemented in C, that performs basic signal
processing on the high-throughput input streams, and
one implemented in Python, that performs direction estimation
at a lower speed.

\begin{subchapter}{\texttt{libsofi.c}}
  As the preprocessing code has to deal with samples
  before the first downsamling stage, the data rates to be
  processed are rather high.
  In order to be able to run on low-range computing hardware
  this part of the processing chain is written in C and
  uses the highly optimized \texttt{fftw3} and \texttt{volk}
  software-libraries, for accelerated \gls{fft} and vector operations. \\

  The program flow starts off by initializing the \gls{sdr}
  dongles. The communication with the dongles is performed
  by using a hardware abstraction layer for software-defined radio
  devices, that was introduced into recent versions of
  the Linux kernel \cite{v4lsdr}. This means, that the program should work
  with all \gls{sdr} devices supported in the Linux kernel,
  as long as their clock sources are synchronized.

  The actual transfer of samples is performed using
  buffers, that are shared between the application and
  the kernel, to minizize the number of
  copy operations performed \cite{v4lmmap}. \\

  To better utilize the processing power of multi-core processors
  \gls{fft} calculations are performed in one thread per
  \gls{sdr} device, distributing the load over the available cores.

  Once the \gls{sdr} devices are initialized and acquiring samples
  they are synchronized by calculating the
  cross-correlation between the per-device streams
  and discarding samples until the peak appears at a
  sample-offset of zero. \\

  Every operations after the synchronization phase is
  performed in the frequency domain. \\

  After the sample-exact synchronization the processing threads, that
  take the time domain samples from the shared buffers and transfer
  them into the frequency domain, are started.

  For every receiver $m$ the \gls{fft} calculation yields phase
  and magnitude information for $1024$ frequency bins.
  The values $i_\text{m,n}$ in these frequency bins can not
  be averaged directly, as the current signal phase can be
  influenced by signal modulation and thus behaves randomly.
  When these stochastic samples are averaged over a sufficiently long
  timespan they would thus cancel out to zero. \\

  In order to be able to perform the necessary averaging
  and downsampling, instead of using the absolute phases
  directly from the \gls{fft}, the phase differences between the receivers
  are calulated prior to the averaging step, using complex conjugate multiplication.
  For four receivers this leads to six difference signals,
  that are calulated according to the equations shown below:

  \begin{equation*}
    \begin{aligned}[c]
      o_\text{1,n}&= i_\text{1,n} \cdot i_\text{2,n}^{\ast} \\
      o_\text{2,n}&= i_\text{1,n} \cdot i_\text{3,n}^{\ast} \\
      o_\text{3,n}&= i_\text{1,n} \cdot i_\text{4,n}^{\ast}
    \end{aligned}
    \qquad \qquad
    \begin{aligned}[c]
      o_\text{4,n}&= i_\text{2,n} \cdot i_\text{3,n}^{\ast} \\
      o_\text{5,n}&= i_\text{2,n} \cdot i_\text{4,n}^{\ast} \\
      o_\text{6,n}&= i_\text{3,n} \cdot i_\text{4,n}^{\ast}
    \end{aligned}
  \end{equation*}

  The resulting values $o_\text{1,n}$ ... $o_\text{6,n}$
  are then averaged over time to reduce the sample-rate
  and forwarded to the next processing step.
\end{subchapter}

\begin{subchapter}{\texttt{libsofi.py}}
  After the downsampling is performed in the backend code
  the data rate is greatly reduced. Further processing
  can thus be performed in Python, an interpreted scripting
  language that, in conjunction with numpy, a software library for
  numeric calulations, makes writing \acrshort{dsp} task easier
  than expressing the algorithms in C. \\

  The data-transfer from C to Python is performed using
  a bit of glue-code, that uses the Python \texttt{ctypes}
  \gls{ffi}.
  This glue-code allows the C backend to be imported
  into the Python code just like a native Python module.
\end{subchapter}

\begin{subchapter}{\texttt{direction.py}}
  The direction estimation code uses the \texttt{numpy} and
  \texttt{scipy} libraries for easy and efficient
  vector operations and signal processing tasks. \\

  This part of the processing chain operates completely
  in the frequency domain, it is responsible for
  performing the compensation tasks to get from the
  raw phase diagram, as seen in Figure \ref{img:annotated_fft_phase_orig},
  to the fully compensated phase diagram,
  as seen in Figure \ref{img:annotated_fft_phase_zoom}.

  It also uses these phase diagrams to calulate direction informations
  for the signal sources in the spectrum. \\

  The compensation works by first finding the local minima in
  the amplitude spectrum, as seen in figure \ref{img:annotated_fft_mag},
  these are the frequencies with the lowest signal strength
  and can thus be assumed to be dominated noise.

  As discussed before noise carries no direction
  information and the phase difference between the
  receivers should thus be, on average, zero.

  The actually measured phase differences at the amplitude minima
  is used as an error input to PID-controllers that
  are configured to compensate for the per-receiver
  mixer phase offsets and sub-sample acuisition
  time differences.
  The controllers constantly optimize the compensation
  parameters to zero out the phase differences at the
  amplitude minima. \\

  After compensation the phase spectrogram looks mostly
  flat with destinct deviations for frequencies
  with strong signals, as can be seen in
  figure \ref{img:annotated_fft_phase_zoom}. \\

  In addition to finding the frequencies, that are dominated
  by noise, the program also finds peaks in received signal strength.
  These peaks correspond to the center frequencies of the currently active
  transmitters.
  These frequencies are then added to a list of sources to be
  analyzed. \\

  As the direction estimation model depends on the
  signal wavelength, the program generates a modeling matrix
  for each of these frequencies. \\

  The code, that generates these matrices, can be seen in
  listing \ref{lst:direction_model}.
  The snippet creates a vector consisting of possible input
  \acrlong{doa} angles (from $-\pi$ to $\pi$ in radians)
  and calculates the expected phase difference for each
  antenna pair based on the antenna pair's orientation,
  distance, input angle and signal wavelength.

  \begin{lstlisting}[language=Python, frame=single,
      numbers=left, label={lst:direction_model},
      caption={Code fragment that calculates the receiver model}]
    # Calculate "relative wavelengths" (in radians)
    # based on the antenna distances in the
    # edge_distances vector and the actual signal wavelength
    rel_wl= 2 * math.pi * edge_distances / wavelength

    # Generate a vector of length self.dir_len and populate
    # it with incidence angles from -pi to pi
    test_angles= np.linspace(-math.pi, math.pi, self.dir_len)

    # Generate an empty target matrix
    pos_mat= np.zeros((self.dir_len, self.edges_count))

    # Go through every incidence angle and populate
    # the corresponding row with the expected phase
    # difference
    for (idx, test_angle) in enumerate(test_angles):
        rel_angles= edge_angles + test_angle
        pos_mat[idx, :]= rel_wl * np.sin(rel_angles)
  \end{lstlisting}

  % VISUAL HACK
  \newpage
  
  Performing a matrix-vector multiplication of this modeling matrix
  and a vector consisting of the measured phase differences
  is equivalent to calculating the scalar dot product of the
  modeled phase difference vector and the actually measured
  phase difference vector for every input angle.
  The code implementing this calculation is
  thus very short, as shown in listing \ref{lst:directioncalc}. \\

  \begin{lstlisting}[language=Python, frame=single,
      numbers=left, label={lst:directioncalc},
      caption={Code fragment that calculates the \gls{doa} estimation}]
    # Get a test matrix according to the
    # wavelength
    pmat= self.get_position_matrix(wl_start, wl_end)

    # Perform a matrix-vector multiplication
    # of the test matrix and the measured phases
    return(pmat @ phase_vector)
  \end{lstlisting}

  As the scalar dot product yields its largest result
  when both vectors are in parallel and its smallest
  result when both vectors are anti-parallel the
  matrix-vector multiplication yields a vector, where
  the position of the largest value corresponds to the
  angle of incidence of the signal.
\end{subchapter}

% VISUAL HACK
\newpage

\begin{subchapter}{\texttt{sofi\_ui.py}}
  The Visualization of the direction estimation is performed
  using the \texttt{matplotlib} and \texttt{Gtk+} software-libraries.

  \figurizefile{diagrams/sofi_ui}
               {img:sofi_ui}
               {The SoFi user interface}
               {0.5}{H}

  Figure \ref{img:sofi_ui} shows the SoFi interface.
  The bottom half displays the phase spectra and
  the top half displays the direction estimation data. \\

  The diameter of the circles in the top plot correspond
  to the signal strengths of the analyzed signals.
  The orientations of the circles indicate the direction
  of arrival of the corresponding signals.
\end{subchapter}
