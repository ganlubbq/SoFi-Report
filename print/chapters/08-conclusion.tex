Unfortunately the project does not perform as
well as expected.

The parts that work as expected are:

\begin{itemize}
\item
  The hardware modifications to lock the input clocks
  to the same source. \\
  There is a remaining slow drift in
  the signal phase that is most likely due to a bit of jitter
  in the LO-frequency generating PLL.
  But due to the drift being slow it can easily be compensated by
  the offset compensation in the frequency domain.

\item
  The sample-exact time compensation works as expected in
  a few iterations.

\item
  The sub-sample time compensation and tuner phase
  offset compensation mostly work.
  When the number of found noise-points upon startup
  is too small the controllers may fail to lock
  correctly as the algorithm only tries to minimize the
  offsets at the noise-points.
  The results of a failed lock can be seen in figure
  \ref{img:failed_lock}.
\end{itemize}

What does not work as expected is the direction
of arrival estimation. Possible reasons may be
multi-path effects in the tested frequency
around $\SI{100}{\mega\hertz}$ or a more
fundamental error in the estimation algorithm
or its implementation.
It is also possible that due to their proximity
the antennas in the array interfere and act
as a single antenna with unkown characteristics\\

To test the DOA-algorithm a simulator
was written that simulates a spectrum
including transmitters and noise but no
multi-pat effects.
The simulator produced samples that were
fed into the SoFi frontend instead of the
SDR samples.
In the simulated case the SoFi DOA estimation
worked as expected indicating that multi-path
effects are indeed a problem or that the simulation
is not accurate enough.
