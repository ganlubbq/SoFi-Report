In the initial SoFi implementation I tried to avoid as many
pitfalls as possible, I stuck to programming languages and
libraries I already knew fairly well and preferred algorithms
and implementations that felt intuitive rather than relying
on well studies approaches presented in academic publications. \\

After gaining some experience using that approach I would
structure a second implementation quite differently. \\

Firstly I would ditch most of the custom C and Python code
and replace most of the signal processing code with implementations
from the GnuRadio framework.

The primary concerns for not choosing GnuRadio in the first
place were lacking support for current versions of Python
(Python 3.X) and a steep learning curve, especially with respect to writing
custom processing steps.
Nevertheless the flexibility of quickly plugging in
filters and other processing steps, that are known to work,
far outweigh the initial time spent getting used to
the framework. \\

Secondly I would introduce a dedicated signal source
for synchronization purposes, possibly a wideband noise
source as demonstrated in rtl\_coherent \cite{rtlcoherent},
as relying on characteristics of background noise
introduces a lot of unkowns, making characterization
of the results and constructing an usefull simulation
quite difficult.

A well characterized reference source would also allow
characterizing the current approach of using background
noise as a means of synchronization by comparing the end-to-end
quality of the \gls{doa} estimation results. \\

I would further use the \gls{music} algorithm in a new implementation,
as it is well studied and known to provide better results
with fewer antennas than more naive algorithms,
like the one used in this thesis \cite{vibergottersten1991}. \\

As working with narrowband signals makes the compensation
of hardware impairments and using \gls{music} easier I
would still employ an \gls{fft}, as shown in
figure \ref{img:fft_music_setup}.

\figurizefile{diagrams/fft_music_setup.tex}
             {img:fft_music_setup}
             {A hybrid \gls{fft} and \gls{music} based approach}
             {0.9}{ht}

The system in the figure uses an array of two antennas and
analyzes the \gls{doa} of signals in four frequency bins. \\

The \gls{fft} is used as a filter bank to
separate the incomming stream into $n$ substreams
of $1/n$ the original sample rate.

As the $n$ substreams are narrowband, the
phase errors introduced by the \gls{losig} phase
and the sub-sample timing offsets can be compensated
by a term $\Delta \varphi(\omega)= \omega \cdot m + b$
as previously discussed in this thesis. \\

For every substream a cross-correlation matrix
of the receivers is calculated and used to estimate
the direction of arrival of the signal in
the corresponding frequency bin by using
the \gls{music} algorithm. \\

The number of frequency bins should be large enough
that one bin is at most occupied by one signal and,
that the narrowband assumtion, when compensation
phase errors, still holds.
