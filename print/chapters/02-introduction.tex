Direction of arrival estimation of \acrshort{rf}-Signals
works by analyzing the signals received by
multiple antennas distributed in space.

The common approach is to exploit the fact,
that the signals received at antennas further
away from a source are more affected by the channel
they are transmitted over.

The effect of the channel can be measured by
observing the signal strength and/or phase
at the receiving antennas.

For systems where the distance between
the sending and the receiving antennas is much
smaller than the distance between the receiving
antennas, the difference in signal strength
are usually very small and using the phase
differences provides better accuracy without
requiring more accurate receivers. \\

The system implemented in this thesis will use \gls{sdr}
technology, fourier transformations and a basic \acrlong{doa}
estimation algorithm to calculate \gls{doa}
information for all signals present in a given frequency range.
More sophisticated algorithms like \acrshort{music} employ
a cross-correlation Matrix as an early processing step,
allowing for better separation of signals and noise
but also limiting the number of signals analyzed at
a time to be less than the number of receivers. \\

Using multiple cheap, independent receivers instead
of purpose-built hardware also poses some challenges
that have to be overcome and will be discussed in
this thesis, like different LO frequencies and differences
in sampling time.

\begin{subchapter}{\gls{doa} principle}
  Figure \ref{img:phased_array} visualizes how signal phases
  can be used to estimate the \acrlong{doa} of a signal
  in the far field. \\

  \figurizefile{diagrams/phased_array.tex}
               {img:phased_array}
               {Array of four antennas with a transmitter to the far right}
               {0.7}{ht}

  The diagram shows four antennas spaced out on a plane
  and a stilized wave that is received by all four of them.
  The source of the wave is assumed to be so far away, that
  the signal-rays are approximately parallel when
  they reach the receivers. \\

  In the depicted constellation the signal has to travel the
  same distance to reach the receivers $A$ and $B$, thus the
  phases at those receivers are the same.
  On the other hand the signal has to travel further to reach
  receiver $D$ than to reach $A$, the difference in phase can
  be directly calculated from the wavelength $\lambda$ of the
  signal and the distance $d_\text{AD}$ between the antennas.

  \begin{equation*}
    \Delta \varphi =  2 \pi \cdot \frac{d_\text{AD}}{\lambda}
  \end{equation*}

  In a more realistic scenario, the the signals will reach the
  receivers at odd angles. Figure \ref{img:angle_two_receivers}
  shows an example where a signal impinges on the receivers at an
  angle of $\alpha=\SI{45}{\degree}$. The length, that can be calculated
  from the phase difference $\Delta \varphi$ between the receivers and
  the wavelength $\lambda$ of the signal, is marked as $\Delta l$.

  The direction of arrival $\alpha$ can then be calculated from
  those lengths using equation \ref{eq:doa_trigonometry}.

  \begin{equation}
    \label{eq:doa_trigonometry}
    \alpha
    = \arccos \left( \frac{\Delta l}{d} \right)
    = \arccos \left( \frac{\Delta \varphi \cdot \lambda}{2 \pi d} \right)
  \end{equation}

  \figurizefile{diagrams/angle_two_receivers.tex}
               {img:angle_two_receivers}
               {Signal arriving at an angle of \SI{45}{\degree}}
               {0.7}{ht}
\end{subchapter}

\begin{subchapter}{Noise influence}
  It is assumed that some of the noises received
  by the receivers stem from external physical
  phenomena.
  These phenomena could be e.g. transients, creating
  wide-band pulses.

  It is also assumed, that the probability
  of these noise-generating events is equal for every direction
  of incidence on the antenna array. \\

  If these events would occur sporadically and the analyzed
  timeframe for each \gls{doa} estimation was small,
  each one of them would yield a distinct \gls{doa} estimation
  result, according to the direction of incidence. \\

  In this thesis the time per \gls{doa} estimation
  is assumed to be large with respect to the mean
  time between these events.
  Thus the direction information results of these
  events can be assumed to cancel out. \\

  A phase difference between two receivers does,
  as shown in the previous section, imply a distinct
  direction of arrival.
  But, as the direction information of the
  noise-signals is expected to cancel out when
  analyzed over a long time-span, the phase differences
  also have to cancel out. \\

  In a competely noise-based signal the measured phase
  differences would thus be, on average, zero.

  In a compelely noise-free signal the phase differences
  would, as shown in the previous section, correspond to the
  antenna distances, wavelength of the signal and the angle
  of incidence.

  It is assumed, that in cases where a signal is
  overlayed by noise, both effects will contribute to the measurement,
  yielding phase differences that are an average of
  the expected value in the noise-free case and the
  expected value (zero) in the noise-only case, weighted by
  the \gls{snr}. \\

  The \gls{doa} estimation algorithm will thus
  have to be able to work with phase differences
  that were scaled by a factor $f$ that depends
  on the \gls{snr}.
  As the \gls{snr} of the closely-spaced antennas
  can be assumed to be approximately the same,
  the factor $f$ can also be assumed to be approximately
  equal for every antenna pair.
\end{subchapter}

\begin{subchapter}{Basic direction estimation}
  As discussed earlier the direction estimation works by
  analyzing differences in signal phase between receivers
  distributed in space. \\

  It is assumed, that the signals $s_0$ ... $s_n$
  can be perfectly separated in the frequency domain,
  meaning that their bandwith is a lot smaller than
  the differences of their center frequencies
  and that the direction of incidence does not
  change rapidly for a fixed frequency. \\

  FM-radio stations fulfill these requirements,
  as they are intended to be easily separable by
  frequency and the (stationary) transmission towers continously
  broadcast at the same frequency. \\

  One example where these assumtions do not hold is
  \acrshort{umts}, where transmitters are not completely
  separable by frequency due to code multiplexing and
  where the locations of active transmitters
  might change quickly due to time-sharing. \\

  The algorithm used in this thesis operates completely
  in the frequency domain, so the time-domain \gls{sdr}
  samples are fourier-transformed, prior to the \gls{doa}
  estimation, using an \gls{fft}.
  The \gls{fft} decomposes the signal into a pre-defined
  number of frequency bins. \\

  The same signal, received at different receivers, will
  be, according to its frequency, sorted into the same
  frequency bin in its respective \gls{fft}.

  The phase information of the signal will be preserved
  when it is transformed by the \gls{fft}.
  This means, that the signal phase can be analyzed by
  analyzing the phase of the corresponding \gls{fft}
  bin.
  The phase difference $\Delta \varphi_{n}$ between two
  receivers for a the $n$th frequency bin can be calulated
  using the complex conjugate
  multiplication, as seen in equation \ref{eq:phccm},
  where $S_\text{1,n}$ and $S_\text{1,n}$ are corresponding
  frequency bins in the \gls{fft} of receivers one and two
  and $A_{n}$ is a real valued scaling factor.

  \begin{equation}
    \label{eq:phccm}
    A_{n} \cdot e^{j \cdot \Delta \varphi_{n}}
    = S_\text{1,n} \cdot S_\text{2,n}^\ast
  \end{equation}

  To estimate the angle of incidence for a signal in a specific
  \gls{fft} bin a vector $\vec{a}_s$ of the calculated phase differences
  for every receiver pair is constructed, where e.g. $\varphi_\text{1,2}$
  is the phase difference between receiver 1 and 2 for that \gls{fft} bin:

  \begin{equation*}
    \vec{a}_s=
    \begin{pmatrix}
      \varphi_\text{1,2} \\
      \varphi_\text{1,3} \\
      \varphi_\text{1,4} \\
      \varphi_\text{2,3} \\
      \varphi_\text{2,4} \\
      \varphi_\text{3,4} \\
    \end{pmatrix}
  \end{equation*}

  Additionally a vector $\vec{b}(\theta)$ is constructed
  that contains the expected phase differences between the
  receivers based on the wavelength $\lambda$, the antenna distance $d$
  and the direction of incidence $\theta$. \\

  The scalar dot product of both vectors $\vec{a}_s$ and $\vec{b}(\theta)$,
  when swept over $\theta$, will have a maximum (at $\theta_\text{max}$)
  when both vectors are parallel and a minimum
  (at $\theta_\text{min}$) when they are anti-parallel. \\

  The vectors $\vec{a}_s$ and $\vec{b}(\theta_\text{max})$ being parallel
  means, that the phase differences in $\vec{a}_s$
  match the expected phases difference for the angle of
  incidence $\theta_\text{max}$.
  Thus $\theta_\text{max}$ is the estimated direction of arrival.
\end{subchapter}

\begin{subchapter}{Receiver offsets: frequency}
  \Gls{sdr} devices are commonly structured as seen in
  figure \ref{img:block_sdr_dongle}, the antenna signal
  $S_\text{in}$ is first amplified by a factor of
  $A_\text{lna}$ by an \acrshort{lna} and then mixed
  down using an \acrshort{iqsig} - \acrshort{losig} signal
  $S_\text{lo}$ to form a complex baseband signal
  $S_\text{bb}$. \\

  \begin{equation}
    \label{eq:lomixer}
    S_\text{bb}
    = S_\text{in} \cdot A_\text{lna} \cdot
      \left( S_\text{lo,i} + j S_\text{lo,q} \right)
    = S_\text{in} \cdot A_\text{lna} \cdot
      e^{j \cdot \left( 2 \pi f_\text{lo} t + \varphi_\text{lo} \right)}
  \end{equation}

  To an user of an \gls{sdr} device the actual waveform
  of the \gls{losig} is usually irrelevant and can be
  modelled as a complex sinusiod of frequency $f_\text{lo}$
  and phase $\varphi_\text{lo}$, as seen in equation \ref{eq:lomixer}. \\

  As discussed earlier \gls{doa} estimation uses the phase
  differences between signals received at different receivers
  to analyze the difference in signal runtime.

  A phase difference between two complex signals can be
  calculated using complex conjugate multiplication as
  shown in \ref{eq:ccmphdiff}.

  \begin{align}
    \label{eq:ccmphdiff}
    A \cdot e^{j \cdot \Delta \varphi}
    &= S_\text{bb,1} \cdot S_\text{bb,2}^\ast \\
    &= S_\text{in,1} \cdot
       e^{j \cdot \left( 2 \pi f_\text{lo,1} t + \varphi_\text{lo,1} \right)}
       \cdot
       S_\text{in,2}^\ast \cdot
       e^{-j \cdot \left( 2 \pi f_\text{lo,2} t + \varphi_\text{lo,2} \right)} \nonumber \\
    &= S_\text{in,1} \cdot S_\text{in,2}^\ast \cdot
       e^{j \cdot \left(
         2 \pi (f_\text{lo,1} - f_\text{lo,2}) t
         + \varphi_\text{lo,1} - \varphi_\text{lo,2}
       \right)} \nonumber \\
    \label{eq:ccmpoff}
    &= S_\text{in,1} \cdot S_\text{in,2}^\ast \cdot
       e^{j \cdot \left(
         2 \pi \Delta f_\text{lo} t
         + \Delta \varphi_\text{lo}
       \right)}
    % Convoluted equations in latex?
    % Still a better read than twilight
  \end{align}

  If the two signals are baseband signals that
  were downconverted using two \gls{losig} signals
  with respective frequencies of $f_\text{lo,1}$ and
  $f_\text{lo,2}$ and phases of $\varphi_\text{lo,1}$
  and $\varphi_\text{lo,2}$ the difference of
  these frequencies and phases will also be present
  in the resulting phase difference $\Delta \varphi$,
  as shown in equation \ref{eq:ccmpoff}. \\

  In order to get the actual phase difference between
  the two input signals $S_\text{in,1}$ and $S_\text{in,2}$
  the \gls{losig} frequencies and phases have to be
  locked ($\Delta f_\text{lo} = 0$, $\Delta \varphi_\text{lo} = 0$).
\end{subchapter}

\begin{subchapter}{Receiver offsets: time}
  Another receiver offset with negative impact on
  \gls{doa} performance is an offset in sampling-time. \\

  If the same signal $s_\text{in}(t)$ is received
  by two receivers $a$ and $b$ the samples wil be
  delayed by the delays $t_\text{a}$ and $t_\text{b}$ before
  being processed to extract phase informations.

  These delays are introduced by unmached cable lengths,
  causing runtime differences, digitizing delays and
  time the samples spend in memory buffers. \\

  When these delayed signals are fourier transformed the
  delays result in rotating signal phases
  \cite[p. 802]{bronstein2016}, as seen in equation
  \ref{eq:tdiffrot}, where $S_\text{in}(j \omega)$ is the fourier
  transformation of $s_\text{in}(t)$.

  \begin{align}
    \label{eq:tdiffrot}
    \mathcal{F} \left\{ s_\text{in}(t - t_\text{a}) \right\}
    = e^{j \omega t_\text{a}} S_\text{in}(j \omega) \\
    \mathcal{F} \left\{ s_\text{in}(t - t_\text{b}) \right\}
    = e^{j \omega t_\text{b}} S_\text{in}(j \omega) \nonumber
  \end{align}

  When the phase difference between these two fourier
  transformed signals is calculated using complex conjugate
  multiplication the difference in the processing delays
  will also manifest in the output phase.

  \begin{align}
    \mathcal{F} \left\{ s_\text{in}(t - t_\text{a}) \right\} \cdot
    \mathcal{F} \left\{ s_\text{in}(t - t_\text{a}) \right\}^\ast
    &=
    e^{ j \omega t_\text{a}} S_\text{in}(j \omega) \cdot
    e^{-j \omega t_\text{b}} S_\text{in}^\ast(j \omega) \\
    &=
    e^{ j \omega \cdot (t_\text{a} - t_\text{b})} \cdot
    S_\text{in}(j \omega) S_\text{in}^\ast(j \omega) \\
    &=
    e^{ j \omega \cdot \Delta t} \cdot
    S_\text{in}(j \omega) S_\text{in}^\ast(j \omega)
  \end{align}

  For small time differences $\Delta t$ the effect on the output
  phases will be a linear slope with respect to the frequency
  $\varphi(\omega)= \omega \cdot c_\text{slope}$ which will wrap
  at $-\pi$ and $\pi$.
  For larger $\Delta t$ the steepness will increase up to a
  point where, due to the wrapping effect, the original phase
  differences will be no longer visible.
\end{subchapter}

\begin{subchapter}{Common noise}
  A common assumption in communications engineering, when
  studying \acrshort{mimo}-systems, is that reciever noises are
  uncorrelated. \\

  This thesis does not use this assumtion but instead
  relies on the presence of a common noise component
  that can be used to synchronize the receivers. \\

  The corresponding noise model is shown in equation
  \ref{eq:commonnoisensig}, where $s_\text{r,1}$ ... $s_\text{r,4}$
  are the receiver signals that are composed of
  the noise-free signals $s_\text{1}$ ... $s_\text{4}$,
  uncorrelated noises $n_\text{1}$ ... $n_\text{4}$
  and a small common noise component $n_\text{c}$.

  \begin{align}
    \label{eq:commonnoisensig}
    s_\text{r,1}(t) &= s_\text{1}(t) + n_\text{1}(t) + n_\text{c}(t) \\
    s_\text{r,2}(t) &= s_\text{2}(t) + n_\text{2}(t) + n_\text{c}(t) \nonumber \\
    s_\text{r,3}(t) &= s_\text{3}(t) + n_\text{3}(t) + n_\text{c}(t) \nonumber \\
    s_\text{r,4}(t) &= s_\text{4}(t) + n_\text{4}(t) + n_\text{c}(t) \nonumber
  \end{align}

  \noindent
  Possible sources of these common noise components in
  an \gls{sdr} system can be:

  \begin{description}
    \item[Passband noise]
      Noise in the desired RF-Band caused by various background-radiation
      sources, received by the antennas.

    \item[Baseband noise]
      Low frequency noise like power-grid radiation or EMI from e.g.
      noisy LED light fixtures that is coupled into the baseband transmission
      lines on the receiver \acrshort{pcb} due to insufficient shielding.

    \item[Common clock noise]
      Noise introduced into the shared clock signal used by the
      receivers.
  \end{description}

  Assuming the signals $s_\text{1}$ ... $s_\text{4}$ in equation
  \ref{eq:commonnoisensig} are filtered out completely the equation
  is simplified as can seen in equation \ref{eq:commonnoise}.

  \begin{align}
    \label{eq:commonnoise}
    n_\text{r,1}(t) &= n_\text{1}(t) + n_\text{c}(t) \\
    n_\text{r,2}(t) &= n_\text{2}(t) + n_\text{c}(t) \nonumber \\
    n_\text{r,3}(t) &= n_\text{3}(t) + n_\text{c}(t) \nonumber \\
    n_\text{r,4}(t) &= n_\text{4}(t) + n_\text{c}(t) \nonumber
  \end{align}

  To prevent the problems stemming from time differences between processed
  signals, discussed in the previous section, the time offsets have
  to be calibrated out. \\

  Assuming every noise component has an autocorrelation that is zero
  for $\tau \neq 0$ and $n_\text{1}$ ... $n_\text{4}$ and $n_\text{c}$
  are uncorrelated, the correlation between
  $n_\text{r,1}$ ... $n_\text{r,4}$ will have a maxmimum at $\tau = 0$. \\

  The time difference between different signals can thus be
  calculated and compensated by filtering out the signals and finding the
  maxmimum correlation between the receiver noises.
\end{subchapter}

\newpage

\begin{subchapter}{\acrshort{music}}
  The direction finding algorithm used in this thesis
  is very basic to simplyfy the implementation.
  There are also a number of more sophisticated algorithms
  that provide better direction estimation results, be it
  by providing better resolution, noise imunity or the
  ability to separate multiple signal sources occupying
  the same frequency band. \\

  On of the earliest of these algorithms is \gls{music},
  first published in 1979 by Ralph O. Schmidt \cite{schmidt1979}. \\

  Unlike the algorithm used in this thesis \gls{music}
  does not perform a transformation to the frequency-domain,
  but performs the direction estimation directly
  in the time-domain. \\

  \begin{subsubchapter}{Signal compositon}
    \gls{music} assumes, that the analyzed spectrum contains
    $I$ discrete, narrowband signal sources overlayed by
    uncorrelated noise.

    A delay in one of these narrowband signals can be approximated
    by a phase shift, as the signal's behaviour closely resembles
    that of a sinusoid \cite[p. 4]{girdmusic}. \\

    In a multi-antenna arrangement the same signal $s_{i}(t)$ will be
    received by the antennas with different amounts of delay.
    Given, that the delays can be modeled by phase shifts,
    the vector of received signals $\vec{x}(t)$ can be expressed
    by the multiplication of the input signal by a vector
    $\vec{a_{i}}(\theta)$ that models the relative phase-shifts.

    \begin{equation}
      \vec{x}(t) =
        \begin{pmatrix}
          e^{-j \cdot \Delta\varphi_{i,1}} \\
          e^{-j \cdot \Delta\varphi_{i,2}} \\
          e^{-j \cdot \Delta\varphi_{i,3}} \\
          e^{-j \cdot \Delta\varphi_{i,4}}
        \end{pmatrix} s_{i}(t) = \\
      \vec{a_{i}}(\theta) s_{i}(t)
    \end{equation}

    The phase-shifting vector $\vec{a_{i}}(\theta)$ is called the
    steering vector, it depends on the geometry of the antenna array
    and the direction of arrival of the signal $i$ \cite{chengokeda2010}. \\

    For multiple signals the model can be expanded by
    introducing a signal vector $\vec{s}(t)$ consisting
    of the $I$ narrowband signals and a steering matrix
    $A$ consisting of the separate steering vectors.

    \begin{equation}
      \vec{x}(t) =
        \begin{pmatrix}
          e^{-j \cdot \Delta\varphi_{1,1}} & e^{-j \cdot \Delta\varphi_{2,1}} & e^{-j \cdot \Delta\varphi_{3,1}} \\
          e^{-j \cdot \Delta\varphi_{1,2}} & e^{-j \cdot \Delta\varphi_{2,2}} & e^{-j \cdot \Delta\varphi_{3,2}} \\
          e^{-j \cdot \Delta\varphi_{1,3}} & e^{-j \cdot \Delta\varphi_{2,3}} & e^{-j \cdot \Delta\varphi_{3,3}} \\
          e^{-j \cdot \Delta\varphi_{1,4}} & e^{-j \cdot \Delta\varphi_{2,4}} & e^{-j \cdot \Delta\varphi_{3,4}}
        \end{pmatrix}
        \cdot
        \begin{pmatrix}
          s_{1}(t) \\
          s_{2}(t) \\
          s_{3}(t)
        \end{pmatrix}
      = \\
      \bm{A} \vec{s}(t)
    \end{equation}

    In an actual system the received signals $\vec{x}(t)$ will also
    each be superimposed by noises $n_{1}$…$n_{4}$.

    \begin{equation}
      \vec{x}(t) =  \bm{A} \vec{s}(t) + \vec{n}(t)
    \end{equation}

    With full knowledge of the steering-matrix $\bm{A}$ the
    directions of arrival could be inferred for every received
    signal. Unfortunately the noise vector $\vec{n}$ and the
    original signals $\vec{s}$ are unkown at the receiver prohibiting
    a direct calculation of the coefficients of $\bm{A}$.
  \end{subsubchapter}

  \begin{subsubchapter}{Signal decomposition}
    To separate the noise and signal subsignals
    MUSIC makes use of a correlation Matrix
    that estimates the correlation between the received
    signals, and principal component analysis to
    extract the main sources ofcorrelation
    \cite{vibergottersten1991} (the signals). \\

    A MUSIC implementation keeps track of the received signal
    correlations $\rho_{i,j}$ in a correlation matrix $R$.

    \begin{equation}
      \bm{R} =
        \begin{pmatrix}
          1          & \rho_{1,2} & \rho_{1,3} & \rho_{1,4} \\
          \rho_{2,1} &          1 & \rho_{2,3} & \rho_{2,4} \\
          \rho_{3,1} & \rho_{3,2} &          1 & \rho_{3,4} \\
          \rho_{4,1} & \rho_{4,2} & \rho_{4,3} &          1 \\
        \end{pmatrix} , \quad \text{where}
        \rho_{i,j} = \frac{\mathrm{Cov}(x_{i}, x_{j})}{\sqrt{\mathrm{Var}(x_{i})\mathrm{Var}(x_{j})}}
      \quad \text{\cite{wikicorrelation}}
    \end{equation}

    As the $R$ matrix is non-negative definite \cite[p. 9]{girdmusic}
    it can be transformed into a diagonal form using
    principal component anaylsis by finding the eigenvectors and -values
    \cite[p. 325]{bronstein2016}.

    \begin{equation}
      \label{eq:pcompdecomp}
      \bm{R} = \bm{U} \bm{D} \bm{U}^{H}
    \end{equation}

    The decomposition of $R$ is shown in equation \ref{eq:pcompdecomp},
    where the diagonal elements of $\bm{D}$
    are the eigenvalues of $\bm{R}$, and $\bm{U}$
    consists of its normalized, corresponding eigenvectors.
    $\bm{D}$ and $\bm{U}$ should be sorted in a way,
    that the eigenvalues are descending in value. \\

    For a number of signals $I$ smaller than the
    number of receivers $N$, the first $I$ eigenvalues
    will correspond to the signal strengths of the signals
    and the remaining $N - I$ eigenvalues will be low as they
    correspond to the correlation of the noise. \\

    This means, that as long as the number of signals is
    known and smaller than the number of receivers the
    diagonal matrix $\bm{D}$ can be separated into a part
    corresponding to the signals (large eigenvalues) and
    a part corresponding to the noise (small eigenvalues).

    MUSIC uses informations from the noise-subspace (small eigenvalues
    and corresponding eigenvectors) to extract
    DOA informations \cite{madisetti2010}, other algorithms,
    like e.g. ESPRIT, use the signal-subspace for DOA-estimation. \\

    To extract direction information from the noise-eigenvectors
    a test steering-vector $\vec{a_{t}}(\theta)$, that contains
    the expected phase-shifts based on the direction of incidence
    $\theta$ and the array geometry, is constructed.

    To calculate the \gls{music} estimation of the signal strength for
    this direction $\theta$ the evaluation function \ref{eq:musicev}
    is used \cite[p. 12]{girdmusic} where $\vec{U}_{n}$ is the $n$th
    (noise-)eigenvector of $R$.

    \begin{equation}
      \label{eq:musicev}
      P(\theta)= \frac{1}{
        \sum_{n=0}^{N-I} \left( \vec{a_{t}}(\theta) \cdot \vec{U}_{n} \right)^2
      }
    \end{equation}

    The results of this evaluation function over the direction
    of incidence $\theta$ form a ``spatial spectrum'', that exhibits
    peaks for directions where a signal is impinging on the array. \\

    This estimation works by exploiting the orthogonality of the
    eigenvectors of $R$ as it implies, that the space spanned by
    the signal eigenvectors has to be orthogonal to the space spanned
    by the noise eigenvectors.
    Thus the scalar dot product of the steering vector $\vec{a_{t}}(\theta)$
    and a noise eigenvector $\vec{U}_{n}$ will be minimal if
    $\vec{a_{t}}(\theta)$ is in the space spanned by the signal
    eigenvectors. \\

    The evaluation function calculates the dot product for every
    noise eigenvector and returns the multiplicative inverse so that a
    $\vec{a_{t}}(\theta)$ in the signal subspace yields a large
    result.
  \end{subsubchapter}
\end{subchapter}
