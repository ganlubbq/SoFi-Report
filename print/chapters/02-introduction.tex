Direction of arrival estimation of RF-Signals
works by analyzing the signals received by
multiple antennas distributed in space.

The common approach is to exploit the fact
that the signals received at antennas further
away from a source are more affected by the channel
they are transmitted over.

The effect of the channel can be measured by
observing the signal strength and/or phase
at the receiving antennas.

For systems where the distance between
the sender and the receiving antennas is much
smaller than the distance between the receiving
antennas the difference in signal strength
are usually very small and using the phase
differences is expected to yield better results. \\

The system that is presented uses Software-defined radio
technology and fourier transformation to calculate DOA
information for all signals present in a given frequency range.
This is a deviation from the more common approach
used in algorithms like MUSIC that use a
cross-correlation Matrix as first processing step and are
thereby limited to analyzing a frequency range that contains
fewer signals than there are receivers. \\

Using multiple cheap, independent receivers also
has some drawbacks that have to be overcome like
different LO frequencies and differences in sampling
time. Solutions to these problems were partly implemented
by modifying the hardware and partly by compensation
in software.
These synchronization techniques will be discussed
in the first few chapters of this thesis.
