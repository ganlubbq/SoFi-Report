The planned goal of this thesis was to explore if
a a low-cost \gls{doa} rig could be built by
utilizing comodity \gls{sdr} receivers
and to analyze if common background-noise
can be used as a sufficient means of
synchronization for \gls{doa} applications. \\

To check these hypotheses a hardware-setup was
assembled, consisting of four frequency-locked
rtl\_sdr receivers and the four corrseponding
antennas spread out on a plane. \\

Software components were written that used
cross-correlations of receiver data-streams
to synchronize samples in time and
performed further compensations to
mitigate hardware impairments stemming from
the independent receivers. \\

Finally a basic \gls{doa} algorithm was implemented
that utilized an \gls{fft} to separate the
analyzed spectrum into multiple frequency bins.
The phase differences of the receivers in these
frequency bins were used directly to derive
the \acrlong{doa} of the impinging signals. \\

Within the processing time of this thesis no
conclusion could be found on whether a
working rig could be implemented with the
specified constraints.

The resolution and accuracy of the measured
\gls{doa} results were rather bad.
It is, as of now, unkown if this was due to
the bad performance of the algorithm used,
other implementation errors or a conceptional
problem of the receivers used. \\

Following are some aspects of the system that
were determined to be working as expected:

\begin{itemize}
  \item
    The hardware modifications to lock the input clocks
    to the same source. \\
    There was a remaining slow drift in
    the signal phase, that was assumed to be most likely due to jitter
    in the \acrshort{losig}-frequency generating \glspl{pll}.
    But, due to this drift being slow, it can easily be compensated by
    the offset-compensating \acrshort{pid}-controllers in the frequency domain.

  \item
    The sample-exact time compensation worked as expected and
    only required a few iterations to sucessfuly lock onto
    the correct time-offset. To within sufficiently few samples.

  \item
    The sub-sample time compensation and tuner-phase
    offset compensation were assumed to be mostly working.
    When the number of found noise-points upon startup
    were too small, the controllers failed to lock
    correctly, as the algorithm only tried to minimize the
    offsets at these noise-points.
    The results of a failed lock can be seen in figure
    \ref{img:failed_lock}.
\end{itemize}

\figurizefile{diagrams/lock_fail}
             {img:failed_lock}
             {A failed sub-sample compensation lock}
             {0.6}{ht}

Possible reasons for the bad \gls{doa} performance include
multi-path effects in the tested frequency range
at about $\SI{100}{\mega\hertz}$, a conceptually bad
performance of the \gls{doa}-estimation algorithm used
or implementation mistakes. \\

It is also possible that, due to their proximity,
the antennas in the array interfered and acted
as a single antenna with unkown and
possibly problematic characteristics. \\

The synchronization by using common background noise
was also not well characterized.
Although the compensated phase diagrams looked well-formed
there might have been a residual systematic
compensation error, distorting the direction estimation.

A setup as demonstrated in rtl\_coherent\cite{rtlcoherent}, that
utilizes a dedicated wideband noise-source, would
have provided a more solid base to independently characterize
the different procesing steps. \\

To test the DOA-algorithm a simulator
was written, that simulated a spectrum
including a few transmitters and independant noise but no
multi-path effects.
The simulator replaced the \gls{sdr} devices in the
SoFi processing chain.

In the simulated case the SoFi DOA estimation
worked as expected, indicating that multi-path
effects were indeed a problem or that the simulation
was otherwise not accurate enough. \\

Due to the simulation only simulating independant
noises, as there was no reliable model for the
assumed common noise term, the sychronization
procedures had to be partially disabled.
The simulation could thus also not be used
to validate the synchronization code. \\
