Unfortunately the project as a whole did not
perform as well as I had hoped it would.

\noindent
The parts that worked as expected were:

\begin{itemize}
  \item
    The hardware modifications to lock the input clocks
    to the same source. \\
    There was a remaining slow drift in
    the signal phase, that was most likely due to a bit of jitter
    in the \acrshort{losig}-frequency generating \glspl{pll}.
    But due to the drift being slow it can easily be compensated by
    the offset compensation \acrshort{pid}-controllers in the frequency domain.

  \item
    The sample-exact time compensation worked as expected and
    only required a few iterations to sucessfuly lock onto
    the correct time-offset.

  \item
    The sub-sample time compensation and tuner phase
    offset compensation mostly worked.
    When the number of found noise-points upon startup
    were too small, the controllers failed to lock
    correctly, as the algorithm only tried to minimize the
    offsets at these noise-points.
    The results of a failed lock can be seen in figure
    \ref{img:failed_lock}.
\end{itemize}

\figurizefile{diagrams/lock_fail}
             {img:failed_lock}
             {A failed sub-sample compensation lock}
             {0.7}{H}

% VISUAL HACK
\newpage

The mayor point that did not work as expected was the direction
of arrival estimation. Possible reasons include
multi-path effects in the tested frequency range
at about $\SI{100}{\mega\hertz}$ or a more
fundamental error in the estimation algorithm
or its implementation. \\

It is also possible that, due to their proximity,
the antennas in the array interfered and acted
as a single antenna with unkown and
possibly problematic characteristics. \\

The synchronization by using common background noise
was also not well characterized.
Although the compensated phase diagrams looked well-formed
there might have been a residual systematic
compensation error distorting the direction estimation.

A setup as demonstrated in rtl\_coherent\cite{rtlcoherent}, that
utilizes a dedicated wideband noise-source, would
have provided a more solid base to independently characterize
the different procesing steps. \\

To test the DOA-algorithm a simulator
was written, that simulated a spectrum
including a few transmitters and independant noise but no
multi-path effects.
The simulator replaced the \gls{sdr} devices in the
SoFi processing chain.

In the simulated case the SoFi DOA estimation
worked as expected, indicating that multi-path
effects were indeed a problem or that the simulation
was otherwise not accurate enough. \\

Due to the simulation only simulating independant
noises, as there was no reliable model for the
assumed common noise term, the sychronization
procedures had to be partially disabled.
The simulation could thus also not be used
to validate the synchronization code. \\
