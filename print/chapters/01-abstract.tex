This thesis aims to explore if a low-cost \gls{doa} estimation
system can be implemented by utilizing commodity \gls{sdr} hardware
and background noise that is common to the different receivers as
a means of time-domain synchronization. \\

The measurement rig is intended to measure the angle of incident
of FM-Radio signals which should, assuming known broadcasting-tower
locations allow estimations of the current receiver location on
a 2D-plane. \\

The rig utilizes DVB-T USB donles that where discovered \cite{rtlhistory}
to contain a feature to pass raw I/Q \acrshort{adc} samples
to the host computer, making it a receive-only \gls{sdr}. \\

Most \gls{doa} algorithms require tight synchronization between
the receiving units in time and frequency. To meet these requirements
\gls{doa} rigs usually employ custom hardware and/or a dedicated
reference signal source. \\

To explore the minimal requirements for a sucessful synchronization
the build at hand uses only minimal hardware modifications
to lock the receiver frequencies to prevent phase drifting. \\

The approach also does not use a dedicated reference signal
for timing synchronization but instead relies on correlations
in the noise-floor of the receivers.
