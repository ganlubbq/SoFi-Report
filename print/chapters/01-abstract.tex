This thesis aims to explore if a low-cost \gls{doa} estimation
system can be implemented by utilizing commodity \gls{sdr} hardware.
And if common background noise at the different receivers can serve
as a sufficient means of time-domain synchronization. \\

The measurement rig is intended to measure the angle of incident
of multiple FM-Radio signals which should, assuming known broadcasting-tower
locations, allow estimations of the current receiver location on
a 2D-plane. \\

The rig utilizes DVB-T USB donles that where discovered \cite{rtlhistory}
to contain a feature to pass raw \acrshort{iqsig} \acrshort{adc} samples
to the host computer, making it a receive-only \gls{sdr}. \\

Most \gls{doa} algorithms require tight synchronization between
the receiving units in time and frequency. To meet these requirements
\gls{doa} rigs usually employ custom hardware and/or dedicated
reference signal sources. \\

To explore the minimal requirements for a sucessful synchronization,
the rig at hand uses only minimal hardware modifications to
the stock USB dongles to lock the receiver
frequencies and to prevent phase drifting. \\

The rig also does not use a dedicated reference signal
for timing synchronization, but instead relies on correlations
in the noise-floor of the receivers to synchronize them
using the noise cross-correlation. \\

By utilizing consumer-grade \gls{sdr} and computing hardware
a \gls{doa}-rig can be assembled for about $\SI{50}{\eurcur}$,
assuming the availability of a sufficiently powerful computer.
