In addition to frequency locking, that is performed in
hardware, additional synchronization steps have to
be performed. These steps are now presented in the
order they are performed. \\

\begin{subchapter}{Sample-exact time compensation}
  The first offset that needs to be compensated stems
  from the fact, that the SDR-dongles do not start
  recording samples at the same time.
  This effect is demonstrated in figures
  \ref{img:samples_actual_time} to \ref{img:samples_synchronized_time}. \\

  The first figure \ref{img:samples_actual_time} shows the
  times at which samples were taken and put into the
  sample-buffer of the corresponding receiver along
  with their index in the sample buffer.
  In a more realisitic diagram the differences between the
  times at which the first samples are taken would be
  a lot bigger compared to the sampling rate as the
  differences in startup time are usually in the range
  of hundreds of milliseconds and the sampling rates are
  in the order of a couple kiloherz to a few megaherz.

  \figurizefile{diagrams/samples_actual_time.tex}
               {img:samples_actual_time}
               {Samples and the actual time they were taken}
               {0.5}{ht}

  Upon reading the samples from the device-buffers
  the information about the absolute aquisition time is
  lost. To a program reading the ADC data the
  samples thus look like in figure \ref{img:samples_observed_time}.

  \figurizefile{diagrams/samples_observed_time.tex}
               {img:samples_observed_time}
               {Sample times when read from the device}
               {0.5}{ht}

  In order to be able to analyze phase differences between
  the signals later on they have to be synchronized to
  a accuracy of a couple sampling times.
  In figure \ref{img:samples_synchronized_time} all
  samles, that where recorded before all ADC where aquiring
  data, are discarded and the input streams are
  synchronized to within one sampling time.

  \figurizefile{diagrams/samples_synchronized_time.tex}
               {img:samples_synchronized_time}
               {Samples after sample-exact synchronization}
               {0.5}{ht}

  In order to perform this sample-exact synchronization
  one of the receivers is used as a reference to which
  the other receivers are synchronized by calculating the
  cross-correlation to the other other receivers and discarding
  samples from the receivers that are lagging behind. \\

  To facilitate calculating the cross correlation over
  more than $100k$ samples a fast fourier transformation
  is used to transfer the calculation into the frequency domain.
  The transformed cross-correlation calculation can be
  seen in equation \ref{eq:fast_cross_correlation}, where
  $\mathcal{F}$ denotes the fourier transformation,
  $\mathcal{F}^{-1}$ is the inverse fourier transformation
  and an asterisk marks the complex conjugate.

  \begin{equation}
    \label{eq:fast_cross_correlation}
    f \star g = \mathcal{F}^{-1}\left(
      \mathcal{F}\left( f \right)^{\ast}
      \cdot
      \mathcal{F}\left( g \right)
    \right)
  \end{equation}

  This compensation step only has to be performed
  once upon startup as the sampling clocks are
  derived via PLL from the same base clock because
  of the hardware modifications.
\end{subchapter}

\begin{subchapter}{Sub-sample time compensation}
  As figure \ref{img:samples_actual_time} shows the
  differences in sample time between the receivers do not
  have to be integer multiples of the sample aquisition durations. \\

  This implies, that the Sample-exact time compensation above
  can not provide complete sample timing compensation as
  a fractional difference still remains afterwards.

  Compensating these offsets in the time domain would
  require fractionally resampling the incoming data streams. \\

  Instead all operations from now on will be performed
  in the frequency domain. Figure \ref{img:preprocessing_chain}
  shows the steps that are performed for every receiver pair prior to
  the sub-sample synchronization. \\

  \figurizefile{diagrams/preprocessing_chain.tex}
               {img:preprocessing_chain}
               {Block diagram of the preporcessing chain for two receivers}
               {0.7}{ht}

  The input streams from the receivers are first delayed
  by the number of samples determined in the previous
  synchronization step so that samples recoreded at the
  same time are in the same position in the steams. \\

  Next the samples are transformed into the time domain
  using fast fourier transformations.
  For one of the output vectors of the two FFTs the
  complex conjugate is calculated element wise. \\

  The resulting frequency-domain vectors are then
  multiplied element wise. This complex conjugate multiplication
  yields a vector of complex numbers which contain the
  phase differences for the observed frequencies in
  their argument. As can be seen in equation
  \ref{eq:multiply_cplx_conj}. \\

  The effect of this operation on the same complex signal
  shifted in phase can be seen in figure \ref{img:complex_conjugate_mul} \\

  In order to reduce the processing load on the following stages
  the resulting complex values are then low pass filtered by
  averaging and subsequently decimated.

  \begin{equation}
    \label{eq:multiply_cplx_conj}
    \begin{split}
      \bm{z}_\text{1} \cdot \bm{z}_\text{2}^{\ast}
      & = \abs{\bm{z}_\text{1}} e^{\arg(\bm{z}_\text{1})}
        \cdot
        \abs{\bm{z}_\text{2}} e^{-\arg(\bm{z}_\text{2})} \\
      & = \abs{\bm{z}_\text{1}} \cdot \abs{\bm{z}_\text{2}}
        e^{\arg(\bm{z}_\text{1}) - \arg(\bm{z}_\text{2})} \\
      & = A^{2} \cdot e^{\Delta \varphi}
    \end{split}
  \end{equation}

  \figurizefile{diagrams/complex_conjugate_mul.tex}
               {img:complex_conjugate_mul}
               {Complex conjugate multiplication of phase shifted signals}
               {0.6}{ht}

  As these operations are performed for every pair
  of antennas in the array, $n_\text{ant}$ input
  signals are mapped to $\left(n_\text{ant}-1\right)!$
  output signals.
  The setup that is analyzed in this thesis uses
  four antennas that produce six difference signal-streams. \\

  In figure \ref{img:annotated_fft_mag} the mean
  absolute values of the resulting complex output
  vectors can be seen. The peaks in the diagram indicate
  the presence of a transmitter at that frequency.

  Figure \ref{img:annotated_fft_phase_orig} shows the phases
  of the six complex output vectors. When correlating the
  diagram with figure \ref{img:annotated_fft_mag} one can already
  see some bumps in the phase differences where there are strong transmitters. \\

  One can also see a rather steep sloping of the phases,
  this is due to the sub-sample timing offsets between the
  ADCs, to compensate them a linear correction is applied.
  The resulting sample time compensated phase diagram can be seen in
  figure \ref{img:annotated_fft_phase_timesync}.

  \figurizefile{diagrams/annotated_fft_phase_orig.tex}
               {img:annotated_fft_phase_orig}
               {Raw phase differences for four receivers}
               {1}{ht}
\end{subchapter}

\begin{subchapter}{Tuner phase offset compensation}
  Figure \ref{img:annotated_fft_phase_timesync} shows a flattened
  phase diagram, but there are still constant offsets.
  These offsets are caused by different phases of the LO-signals
  that are used for mixing down the RF-signals in the tuners. \\

  Figure \ref{img:annotated_fft_phase_comp} shows the phases
  after compensating for those offsets. Frequency ranges where
  the noise is dominant show no phase differences between
  the receivers, while ranges with active transmitters contain
  visible bumps. When zoomed in like in figure \ref{img:annotated_fft_phase_zoom}
  these bumps become a lot more prominent. \\

  The relative phase differences will later be used to
  determine the direction of the incoming signal.
\end{subchapter}
